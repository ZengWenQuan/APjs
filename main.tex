%% Beginning of file 'sample7.tex'
%%
%% Version 7. Created January 2025.  
%%
%% AASTeX v7 calls the following external packages:
%% times, hyperref, ifthen, hyphens, longtable, xcolor, 
%% bookmarks, array, rotating, ulem, and lineno 
%%
%% RevTeX is no longer used in AASTeX v7.
%%
\documentclass[linenumbers,trackchanges]{aastex7}
% 使用 ctex 宏包来支持中文排版,包括对字体、标点等的支持。
\usepackage{ctex}

% 提供了制作高质量表格的功能,如 \toprule, \midrule 和 \bottomrule 等命令,用于美化表格。
\usepackage{booktabs}

% 用于插入图像文件,支持多种格式(如 PDF、PNG、JPG 等),并提供了调整图像大小、旋转等功能。
\usepackage{graphicx} 
%\graphicspath{{figures/}} % 设置图片所在的目录,方便引用时指定路径。

% amsmath 宏包增强了 LaTeX 的数学功能,提供了一系列新命令和环境来简化数学公式的输入与排版。
\usepackage{amsmath}

% lastpage 宏包允许用户在文档中引用总页数,常用于页脚或页眉中显示“第 X 页,共 Y 页”。
\usepackage{lastpage}%获得总页数

% fancyhdr 宏包用于定制页眉和页脚,提供了灵活的方式来设置文本样式、页码位置等。
\usepackage{fancyhdr}

% float 宏包增强了浮动体(如图形和表格)的控制能力,通过新增选项如 [H] 来强制放置位置。
\usepackage{float}

% biblatex 宏包提供了一个强大且灵活的参考文献管理解决方案,支持多种引用样式和数据库格式。
\usepackage{biblatex}
\addbibresource{references.bib}


%%
%% This initial command takes arguments that can be used to easily modify 
%% the output of the compiled manuscript. Any combination of arguments can be 
%% invoked like this:
%%
%% \documentclass[argument1,argument2,argument3,...]{aastex7}
%%
%% Six of the arguments are typestting options. They are:
%%
%%  twocolumn   : two text columns, 10 point font, single spaced article.
%%                This is the most compact and represent the final published
%%                derived PDF copy of the accepted manuscript from the publisher
%%  default     : one text column, 10 point font, single spaced (default).
%%  manuscript  : one text column, 12 point font, double spaced article.
%%  preprint    : one text column, 12 point font, single spaced article.  
%%  preprint2   : two text columns, 12 point font, single spaced article.
%%  modern      : a stylish, single text column, 12 point font, article with
%% 		  wider left and right margins. This uses the Daniel
%% 		  Foreman-Mackey and David Hogg design.
%%
%% Note that you can submit to the AAS Journals in any of these 6 styles.
%%
%% There are other optional arguments one can invoke to allow other stylistic
%% actions. The available options are:
%%
%%   astrosymb    : Loads Astrosymb font and define \astrocommands. 
%%   tighten      : Makes baselineskip slightly smaller, only works with 
%%                  the twocolumn substyle.
%%   times        : uses times font instead of the default.
%%   linenumbers  : turn on linenumbering. Note this is mandatory for AAS
%%                  Journal submissions and revisions.
%%   trackchanges : Shows added text in bold.
%%   longauthor   : Do not use the more compressed footnote style (default) for 
%%                  the author/collaboration/affiliations. Instead print all
%%                  affiliation information after each name. Creates a much 
%%                  longer author list but may be desirable for short 
%%                  author papers.
%% twocolappendix : make 2 column appendix.
%%   anonymous    : Do not show the authors, affiliations, acknowledgments,
%%                  and author contributions for dual anonymous review.
%%  resetfootnote : Reset footnotes to 1 in the body of the manuscript.
%%                  Useful when there are a lot of authors and affiliations
%%		    in the front matter.
%%   longbib      : Print article titles in the references. This option
%% 		    is mandatory for PSJ manuscripts.
%%
%% Since v6, AASTeX has included \hyperref support. While we have built in 
%% specific %% defaults into the classfile you can manually override them 
%% with the \hypersetup command. For example,
%%
%% \hypersetup{linkcolor=red,citecolor=green,filecolor=cyan,urlcolor=magenta}
%%
%% will change the color of the internal links to red, the links to the
%% bibliography to green, the file links to cyan, and the external links to
%% magenta. Additional information on \hyperref options can be found here:
%% https://www.tug.org/applications/hyperref/manual.html#x1-40003
%%
%% The "bookmarks" has been changed to "true" in hyperref
%% to improve the accessibility of the compiled pdf file.
%%
%% If you want to create your own macros, you can do so
%% using \newcommand. Your macros should appear before
%% the \begin{document} command.
%%
\newcommand{\vdag}{(v)^\dagger}
\newcommand\aastex{AAS\TeX}
\newcommand\latex{La\TeX}
%%%%%%%%%%%%%%%%%%%%%%%%%%%%%%%%%%%%%%%%%%%%%%%%%%%%%%%%%%%%%%%%%%%%%%%%%%%%%%%%
%%
%% The following section outlines numerous optional output that
%% can be displayed in the front matter or as running meta-data.
%%
%% Running header information. A short title on odd pages and 
%% short author list on even pages. Note that this
%% information may be modified in production.
%%\shorttitle{AASTeX v7 Sample article}
%%\shortauthors{The Terra Mater collaboration}
%%
%% Include dates for submitted, revised, and accepted.
%%\received{February 1, 2025}
%%\revised{March 1, 2025}
%%\accepted{\today}
%%
%% Indicate AAS Journal the manuscript was submitted to.
%%\submitjournal{PSJ}
%% Note that this command adds "Submitted to " the argument.
%%
%% You can add a light gray and diagonal water-mark to the first page 
%% with this command:
%% \watermark{text}
%% where "text", e.g. DRAFT, is the text to appear.  If the text is 
%% long you can control the water-mark size with:
%% \setwatermarkfontsize{dimension}
%% where dimension is any recognized LaTeX dimension, e.g. pt, in, etc.
%%%%%%%%%%%%%%%%%%%%%%%%%%%%%%%%%%%%%%%%%%%%%%%%%%%%%%%%%%%%%%%%%%%%%%%%%%%%%%%%
%%
%% Use this command to indicate a subdirectory where figures are located.
%%\graphicspath{{./}{figures/}}
%% This is the end of the preamble.  Indicate the beginning of the
%% manuscript itself with \begin{document}.

\begin{document}

\title{Template \aastex v7 Article with Examples\footnote{Footnotes can be added to titles}}

%% A significant change from AASTeX v6+ is in the author blocks. Now an email
%% address is required for each author. This means that each author requires
%% at least one of the following:
%%
%% \author
%% \affiliation
%% \email
%%
%% If these three commands are not available for each author, the latex
%% compiler will issue an error and if you force the latex compiler to continue,
%% it will generate an incomplete pdf.
%%
%% Multiple \affiliation commands are allowed and authors can also include
%% an optional \altaffiliation to indicate a status, i.e. Hubble Fellow. 
%% while affiliations are indexed as footnotes, altaffiliations are noted with
%% with a non-numeric footnote that is set away from the numeric \affiliation 
%% footnotes. NOTE that if an \altaffiliation command is used it must 
%% come BEFORE the \affiliation call, right after the \author command, in 
%% order to place the footnotes in the proper location. Because non-numeric
%% symbols are used, \altaffiliation should be used sparingly.
%%
%% In v7 the \author command takes an optional argument which provides 
%% additional metadata about the author. Authors can provide the 16 digit 
%% ORCID, the surname (family or last) name, the given (first or fore-) name, 
%% and a name suffix, e.g. "Jr.". The syntax is:
%%
%% \author[orcid=0000-0002-9072-1121,gname=Gregory,sname=Schwarz]{Greg Schwarz}
%%
%% This name metadata in not shown, it is only for parsing by the peer review
%% system so authors can be more easily identified. This name information will
%% also be sent to the publisher so they can include it in the CROSSREF 
%% metadata. Including an orcid will hyperlink the author name to the 
%% author's ORCID page. Note that  during compilation, LaTeX will do some 
%% limited checking of the format of the ID to make sure it is valid. If 
%% the "orcid-ID.png" image file is  present or in the LaTeX pathway, the 
%% ORCID icon will appear next to the authors name.
%%
%% Even though emails are now required for each author, the \email does not
%% produce output in the compiled manuscript unless the optional "show" command
%% is used. For example,
%%
%% \email[show]{greg.schwarz@aas.org}
%%
%% All "shown" emails are show in the bottom left of the first page. Due to
%% space constraints, only a few emails should be shown. 
%%
%% To identify a corresponding author, use the \correspondingauthor command.
%% The command appends "Corresponding Author: " to the argument it appears at
%% the bottom left of the first page like the output from \email. 

\author[orcid=0000-0000-0000-0001,sname='North America']{Tundra North America}
\altaffiliation{Kitt Peak National Observatory}
\affiliation{University of Saskatchewan}
\email[show]{fakeemail1@google.com}  

\author[orcid=0000-0000-0000-0002,gname=Bosque, sname='Sur America']{Forrest Sur Am\'{e}rica} 
\altaffiliation{Las Campanas Observatory}
\affiliation{Universidad de Chile, Department of Astronomy}
\email{fakeemail2@google.com}

\author[gname=Savannah,sname=Africa]{S. Africa}
\affiliation{South African Astronomical Observatory}
\affiliation{University of Cape Town, Department of Astronomy}
\email{fakeemail3@google.com}

\author{River Europe}
\affiliation{University of Heidelberg}
\email{fakeemail4@google.com}

\author[0000-0000-0000-0003,sname=Asia,gname=Mountain]{Asia Mountain}
\altaffiliation{Astrosat Post-Doctoral Fellow}
\affiliation{Tata Institute of Fundamental Research, Department of Astronomy}
\email{fakeemail5@google.com}

\author[0000-0000-0000-0004]{Coral Australia}
\affiliation{James Cook University, Department of Physics}
\email{fakeemail6@google.com}

\author[gname=IceSheet]{Penguin Antarctica}
\affiliation{Amundsen–Scott South Pole Station}
\email{fakeemail7@google.com}

\collaboration{all}{The Terra Mater collaboration}

%% Use the \collaboration command to identify collaborations. This command
%% takes an optional argument that is either a number or the word "all"
%% which tells the compiler how many of the authors above the command to
%% show. For example "\collaboration[all]{(DELVE Collaboration)}" wil include
%% all the authors above this command.
%%
%% Mark off the abstract in the ``abstract'' environment. 
\begin{abstract}

This example manuscript is intended to serve as a tutorial and template for
authors to use when writing their own AAS Journal articles. The manuscript
includes a history of \aastex\ and documents the new features in the
previous versions as well as the new features in version 7. This
manuscript includes many figure and table examples to illustrate these new
features.  Information on features not explicitly mentioned in the article
can be viewed in the manuscript comments or more extensive online
documentation. Authors are welcome replace the text, tables, figures, and
bibliography with their own and submit the resulting manuscript to the AAS
Journals peer review system.  The first lesson in the tutorial is to remind
authors that the AAS Journals, the Astrophysical Journal (ApJ), the
Astrophysical Journal Letters (ApJL), the Astronomical Journal (AJ), and 
the Planetary Science Journal (PSJ) all have a 250 word limit for the 
abstract. The limit is 150 for RNAAS manuscripts. If you exceed this length 
the Editorial office will ask you to shorten it. This abstract has 189 words.

\end{abstract}

%% Keywords should appear after the \end{abstract} command. 
%% The AAS Journals now uses Unified Astronomy Thesaurus (UAT) concepts:
%% https://astrothesaurus.org
%% You will be asked to selected these concepts during the submission process
%% but this old "keyword" functionality is maintained in case authors want
%% to include these concepts in their preprints.
%%
%% You can use the \uat command to link your UAT concepts back its source.
\keywords{\uat{Galaxies}{573} --- \uat{Cosmology}{343} --- \uat{High Energy astrophysics}{739} --- \uat{Interstellar medium}{847} --- \uat{Stellar astronomy}{1583} --- \uat{Solar physics}{1476}}

%% From the front matter, we move on to the body of the paper.
%% Sections are demarcated by \section and \subsection, respectively.
%% Observe the use of the LaTeX \label
%% command after the \subsection to give a symbolic KEY to the
%% subsection for cross-referencing in a \ref command.
%% You can use LaTeX's \ref and \label commands to keep track of
%% cross-references to sections, equations, tables, and figures.
%% That way, if you change the order of any elements, LaTeX will
%% automatically renumber them.

\section{Introduction}
CEMP(Carbon-Enhanced Metal-Poor)恒星是一类贫金属、富碳的老年恒星,其金属丰度([Fe/H])通常低于太阳值的千分之一,而碳丰度却显著增强。它们被认为是研究宇宙早期化学演化和恒星形成过程的关键天体。根据碳 enrich 水平和重元素特征,CEMP恒星被进一步划分为多个子类,如CEMP-s(s-process元素丰富)、CEMP-r(r-process元素丰富)和CEMP-no(氮和氧也贫化)。这些子类反映了不同的起源机制。

CEMP-s恒星约占CEMP恒星的80\%,多位于双星系统中,其碳和s-process元素 enrichment 可能源自伴星——一颗已经演化的AGB(渐近巨星支)恒星通过质量转移污染了原初气体 \cite{abia2002}。这一模型得到了观测到的双星轨道周期和自转速度的支持。

CEMP-no恒星则更为古老,可能形成于首批恒星(Population III)爆炸后的遗迹气体中。这些气体富含碳但缺乏 heavier metals,因此形成的第二代恒星继承了这种成分 \cite{christlieb2002}。例如,HE 1327-2326 是已知最metal-poor的CEMP-no恒星之一,其大气参数和化学组成为此假说提供了直接证据 \cite{ito2009}。

CEMP-r恒星含有较高的r-process元素(如铕Eu),这表明它们的形成环境受到了中子星并合或磁驱动超新星等事件的影响 \cite{hansen2011}。然而,这类恒星较为罕见,其确切起源仍需更多观测数据验证。

CEMP-no恒星多见于极低[Fe/H](< -3),如HE 1327-2326 ([Fe/H] ≈ -5.7, \cite{ito2009}) 和 SMSS J0313-6708 ([Fe/H] < -7.1, \cite{keller2014}),这表明它们可能形成于首批恒星(Population III)超新星爆炸后富含碳但贫metal的气体中。相比之下,CEMP-s恒星的[Fe/H]相对较高(≈-2.5至-3.0),这可能与双星系统中AGB伴星的质量转移有关,该过程需足够的时间积累,故对应稍高的metallicity (\cite{lucatello2005}).

CEMP-no恒星的[C/Fe]通常超过+1.0,且在极低[Fe/H]下保持高位,暗示其碳源可能来自高能爆发现象(如 faint supernovae)。而CEMP-s恒星的[C/Fe]虽同样高,但常伴随s-process元素(如Ba),其形成与AGB星的慢中子俘获过程相关,且在[Fe/H] > -3时更为常见 (\cite{abia2002}, \cite{bisterzo2011}).

CEMP-no恒星常表现为低[N/Fe]和高[C/N],与Population III恒星的混合模型一致 (\cite{norris2013}). CEMP-s恒星则多有[Ba/Fe] > +1.0,且[Ba/Eu]接近太阳能值,指示s-process主导 (\cite{masseron2010}). 少数CEMP-r恒星同时具有r-process元素(如Eu) enrichment,可能由中子星并合或磁驱动超新星引起 (\cite{hansen2011}).

观测数据显示,当[Fe/H] < -3时,CEMP恒星比例显著上升(约20\%),其中CEMP-no占主导 (\cite{yoon2016}). 这一趋势符合理论预测:早期宇宙中低metallicity气体更易在碳增强条件下形成低质量恒星 (\cite{frebel2015}).

CEMP恒星的研究不仅有助于理解早期宇宙的化学增丰历史,还为恒星形成和核合成理论提供了宝贵约束。未来高分辨率光谱观测和大样本统计将进一步揭示各类CEMP恒星的形成机制及其宇宙学意义。

我们对恒星的所有认知,从化学成分到物理状态,几乎都源于对其光谱的精细解读。理论上,恒星光谱由两部分组成:一个由其炽热表面发出的、平滑的**连续谱**作为背景;以及叠加其上的、由恒星外层大气中各种元素吸收特定波长光子而形成的**吸收线**。这些吸收线如同一张独特的“条形码”,其位置、深度和形状精确地揭示了恒星的化学丰度(如[Fe/H]和[C/Fe])、温度、压力乃至自转速度。然而,从观测数据中准确提取这些微弱的物理信号并非易事,需要仔细区分真实的吸收线与宇宙线、探测器噪声等干扰,这是恒星光谱分析中的关键挑战。

CEMP(Carbon-Enhanced Metal-Poor)恒星作为宇宙早期化学演化的关键探针,其发现与研究离不开大型天文巡天项目的推动。LAMOST(郭守敬望远镜)凭借其高效的光纤光谱观测能力,已成为发现和表征CEMP恒星的重要平台。自运行以来,LAMOST已释放数百万条恒星光谱数据,极大拓展了CEMP恒星样本量。例如,Li et al. (2015)\cite{li2015} 利用LAMOST数据鉴定出逾千颗CEMP候选星,显著提升了该类天体的统计基础。此外,结合Gaia的距离和自行数据及SDSS的测光信息,研究人员得以进一步解析CEMP恒星的大气参数和 kinematic 特性,深入探讨其起源机制(Yuan et al., 2020)\cite{yuan2020}。未来,LAMOST将持续助力CEMP恒星的系统性研究,为理解首批恒星核合成及银河系形成提供宝贵线索。


\section{相关工作和方法}

在利用大规模巡天数据搜寻如CEMP星等特殊恒星时,模板匹配法是一种被广泛应用的经典技术。其核心思想是通过将观测光谱与一个覆盖了广阔参数空间($T_{\text{eff}}$, $\log g$, [Fe/H], [C/H]等)的参考光谱库进行比对,通过寻找最佳匹配来推断前者的物理参数。针对LAMOST巡天数据,该方法的具体实施包含以下三个关键环节:
\subsection{模板匹配法}
\subsubsection{参考光谱库的构建}
模板库的完备性与精度是决定该方法成败的基石。一个为CEMP星搜索优化的模板库需满足以下要求:
\begin{itemize}
    \item \textbf{理论光谱的生成:} 使用如MOOG等合成光谱代码,结合MARCS等先进的恒星大气模型,生成覆盖CEMP星典型参数范围的理论光谱。在生成过程中,必须精细考虑关键分子带(如CH G-band, C$_2$ Swan bands)的贡献,并模拟LAMOST的仪器展宽和星际消光效应。
    \item \textbf{参数空间的覆盖:} 模板库必须在$T_{\text{eff}}$ (4000-6500 K), $\log g$ (1.0-4.5 dex), [Fe/H] (-4.0 to 0.0 dex), 和 [C/Fe] (0.0 to +3.0 dex) 等维度上进行精细网格化采样,以确保对各类恒星都有精确的模板对应。
    \item \textbf{经验光谱的补充:} 为修正理论模型在极低金属丰度区域可能存在的系统偏差,通常会引入来自高分辨率观测(如SDSS/SEGUE, HET/HRS)的真实CEMP星光谱作为经验模板,以提升模板库的可靠性。
\end{itemize}

\subsubsection{光谱匹配与参数优化}
此环节的目标是在模板库中为每一条观测光谱找到最佳匹配。对于经过归一化等预处理的LAMOST低分辨率光谱,该过程通常通过$\chi^2$最小化来实现。具体而言,即求解一个非线性优化问题,其目标函数如下:
$
\min_{\theta, a, v_{\text{rad}}} \sum_{i} \left[ \frac{f_{\text{obs}}(\lambda_i) - a \cdot T(\lambda_i(1+v_{\text{rad}}/c); \theta)}{\sigma_i} \right]^2
$
其中,$f_{\text{obs}}$是观测流量,$T(\lambda; \theta)$是参数为$\theta$的模板光谱,$a$和$v_{\text{rad}}$分别是归一化尺度因子和视向速度,$\sigma_i$是噪声。此优化过程通常采用Levenberg-Marquardt等高效算法并行求解。参数的最终误差则常通过MCMC等贝叶斯方法进行估计。

\subsubsection{结果的后处理与验证}
为确保结果的可靠性,必须对初步匹配结果进行严格的后处理与验证:
\begin{itemize}
    \item \textbf{关键谱线检验:} 对所有候选体,计算关键的碳分子带(如CH G-band)的等值宽度或线指数,并设置信噪比阈值,以剔除因噪声导致的伪信号。
    \item \textbf{人工目视检查:} 对所有高置信度的候选体进行人工检查,以排除宇宙射线、光纤交叉污染等仪器效应的干扰。
    \item \textbf{外部交叉验证:} 将得到的结果与来自其他高分辨率巡天(如APOGEE)的参数测量结果进行比对,以评估系统误差和测量精度。
\end{itemize}

尽管模板匹配法在处理低信噪比数据时表现稳健,并能同时给出多个恒星参数,但其性能高度依赖于模板库的完备性,且在面对千万量级的巡天数据时,其计算成本高昂,难以满足自动化、高效率分析的需求。



随着LAMOST等大规模光谱巡天项目的深入,数据量呈爆炸式增长,传统的模板匹配法在效率和可扩展性上逐渐暴露瓶颈。与此同时,以支持向量机、随机森林为代表的传统机器学习方法,虽在一定程度上提升了自动化水平,但其性能高度依赖于人工设计的特征(如等值宽度、颜色指数等),在面对高维、非线性的光谱数据时,其建模能力和泛化性均受到限制。

近年来,数据驱动的深度学习方法为恒星光谱分析带来了范式上的革新。这类方法能够直接从原始光谱数据中自动学习层次化的特征表示,极大地降低了对特征工程的依赖,并在参数估计的精度和效率上取得了突破。在恒星参数估计领域,已有多种深度学习模型被成功应用:

\subsubsection{The Cannon: 数据驱动的光谱建模基石}
\label{subsec:cannon}
由Ness等人提出的\textbf{The Cannon} \cite{ness2015cannon}是该领域的里程碑式工作。它并非一个物理模型,而是一个数据驱动的生成模型。其核心思想是:在存在一个高精度“训练集”(即一组同时拥有光谱和精确参数的恒星)的前提下,可以学习到一个普适的函数,该函数将任意恒星的光谱流量与它的物理参数(“标签”)精确地联系起来。\textit{The Cannon}通常使用一个简单的多项式模型来拟合这种关系。在训练阶段,模型从参考样本中学习到光谱流量与恒星标签之间的映射关系;在应用阶段,对于只有光谱的未知天体,模型能够以极高的速度反解出其最可能的标签。该方法以其高效、稳健以及在低信噪比数据上的优异表现,被广泛应用于APOGEE等大型巡天项目中。

\subsubsection{SPCANet: 融合主成分分析的混合模型}
\textbf{SPCANet} \cite{wang2020spcanet}是一种结合了主成分分析(PCA)与深度神经网络(DNN)的混合模型。它首先利用PCA对高维光谱数据进行降维,提取出最主要的成分作为特征,此举能有效去除噪声并减少信息冗余。随后,这些低维特征被送入一个深度神经网络,通过非线性回归建模,学习从光谱主成分到恒星物理参数的复杂映射。SPCANet结合了PCA的线性和DNN的非线性建模能力,在处理低信噪比光谱时表现出很强的鲁棒性。

\subsubsection{SLAM: 端到端的卷积神经网络方法}
\textbf{SLAM (Stellar LAbel Machine)} \cite{zhang2020slam}是专为大规模光谱巡天设计的端到端深度学习模型。它采用卷积神经网络(CNN)架构,能够直接从原始的一维光谱像素中自动学习相关特征,无需任何人工干预。通过多层卷积与池化操作,SLAM能够有效捕捉光谱中的局部和全局结构信息(如吸收线的轮廓、相对强度等),并最终通过全连接层回归得到恒星参数。SLAM的端到端特性使其具备极高的自动化处理能力和对数据噪声、采样不均等问题的鲁棒性,是处理海量光谱数据的理想选择。

\subsubsection{SpecCLIP: 基于对比学习的光谱基础模型}近年来,天文学界开始借鉴自然语言处理和计算机视觉领域的成功经验,发展面向海量多模态数据的“基础模型”(Foundation Model)。\textbf{SpecCLIP} \cite{li2023, zhang2024} 正是这一趋势下的前沿代表,它将恒星光谱视为一种结构化的“语言”,旨在学习到一种能够泛化至多种下游任务的普适性光谱表示。SpecCLIP的核心思想源于强大的`CLIP` (Contrastive Language-Image Pre-training) 框架。它创新性地采用**对比学习**策略,将来自不同巡天项目、具有不同特征(如分辨率、波长覆盖)的光谱数据进行对齐。例如,将LAMOST的低分辨率光谱与Gaia BP/RP (XP)的光谱在同一个高维“嵌入空间”(Embedding Space)中进行匹配。其目标是让来自同一颗恒星的不同光谱在该空间中的表示尽可能接近,而来自不同恒星的光谱则相互远离。这种自监督的学习方式使得模型能够从海量的无标签数据中学习到光谱的内在物理结构。作为一个多模态模型,SpecCLIP能够自然地融合光谱信息与Gaia提供的高精度天体测量数据(如视差、自行)。这种跨模态学习带来了显著优势:一方面,精确的距离信息可以有效打破恒星参数(如$T_{\text{eff}}$和$\log g$)之间的简并性,从而大幅提升参数估计的精度;另一方面,统一的嵌入空间使得在不同巡天数据之间进行交叉定标和信息迁移成为可能。SpecCLIP的应用远不止于高精度的参数估计。其强大的表示学习能力使其在**异常光谱检测**、**相似天体搜寻**以及**跨模态光谱生成**(例如,从Gaia XP光谱预测LAMOST光谱)等任务中也展现出巨大潜力,为天文学研究提供了全新的探索工具。

\subsection{研究动机与方法革新}
\label{subsec:motivation_dl}

\textbf{下一代光谱巡天引发的数据挑战与科学机遇。}以LAMOST和SDSS-V为代表的光谱巡天项目,正将天文学带入一个前所未有的“太字节”(Petabyte)时代。其产生的数据量已逾千万量级,并仍在持续增长。这一数据的“洪流”为在银河系中系统性地搜寻像CEMP星这样的稀有天体提供了空前的机遇,但同时也对传统分析方法构成了严峻挑战。如前所述,经典的模板匹配法(第\ref{subsec:template_cemp}节)在这一新形势下面临着三个难以逾越的瓶颈:

\begin{itemize}
    \item \textbf{计算可扩展性瓶颈:} 模板匹配本质上是基于$\chi^2$最小化的网格搜索,其计算成本随参数空间维度(如考虑C, N, $\alpha$等多种元素丰度)呈指数级增长,无法满足海量数据的实时处理需求。
    \item \textbf{低信噪比下的稳健性问题:} 对于LAMOST产生的大量低信噪比(S/N < 15)光谱,关键的吸收线特征(如CH分子带)极易被噪声淹没,导致模板匹配法的参数测定,特别是碳丰度的误差急剧增大,可靠性降低。
    \item \textbf{模板库的不完备性与泛化难题:} 任何理论或经验模板库都难以完全覆盖所有可能的恒星参数,尤其是在极端物理区域(如[Fe/H] < -3.5或[C/Fe] > +2.0)。这使得模板匹配法在搜寻最稀有、最极端的CEMP星时,其探测效率和可靠性都受到根本性限制。
\end{itemize}

\textbf{深度学习:应对挑战的系统性解决方案。} 近年来,深度学习的迅猛发展为突破上述瓶颈提供了全新的、系统性的解决思路。它不再依赖于固定的模板库,而是直接从数据中学习光谱与物理参数之间的复杂映射关系。具体而言:

\begin{itemize}
    \item \textbf{应对可扩展性瓶颈:} 以卷积神经网络(CNN)为代表的端到端模型 \cite{ting2019},能够直接从原始光谱像素中自动提取特征并进行回归,完全绕开了计算昂贵的网格搜索,处理速度相比传统方法可提升数个数量级。
    \item \textbf{应对低信噪比挑战:} CNN的层次化特征提取能力使其对噪声具有更强的鲁棒性。同时,贝叶斯神经网络等技术的发展 \cite{bayescnn2022},使得模型在给出参数预测的同时,还能提供可靠的不确定性量化,这对于评估低信噪比数据的分析结果至关重要。
    \item \textbf{应对泛化能力难题:} 生成对抗网络(GAN)等生成式模型 \cite{zhang2021_gan},能够学习真实光谱的数据分布,并据此生成海量的、物理真实的合成光谱,从而有效扩充和补完训练集在极端参数区域的覆盖,极大提升了模型在稀有天体搜寻任务中的泛化能力。
\end{itemize}

正是基于以上挑战与机遇,本研究旨在开发一种专为大规模巡天数据中的CEMP星高效搜寻而设计的新型深度学习框架。我们的方法将融合上述先进技术,以期在保证高精度的前提下,实现对海量光谱数据快速、稳健的自动化分析,从而为揭示银河系早期的化学演化历史提供一个前所未有的大样本。

\subsection{研究方法概述}
CEMP(Carbon-Enhanced Metal-Poor)恒星是研究早期宇宙化学演化和第一代恒星形成的关键探针。其分类标准随着观测技术的进步和理论模型的发展而不断演进\cite{beers1992, aoki2007, norris2013}。本研究采纳当前广泛接受的严格标准(详见表\ref{tab:star_criteria}),基于大规模巡天数据,利用数据驱动的方法系统性地搜寻CEMP恒星候选体。

我们的研究流程遵循一个三阶段框架,旨在从海量低分辨率光谱数据中高效、准确地识别目标。具体步骤如下:
\begin{enumerate}
    \item \textbf{构建训练样本库:} 实验数据源于LAMOST DR11发布的低分辨率光谱。为给后续深度学习模型提供精确的标签,我们首先将LAMOST星表与多个高分辨率光谱巡天(如APOGEE DR17、LAMOST-Subaru等)的星表进行交叉匹配。匹配过程基于天球坐标(赤经RA, 赤纬Dec),容许的最大位置误差为3角秒,最终建立一个包含低分辨率光谱和高精度恒星物理参数(有效温度$T_{\text{eff}}$、表面重力$\log g$、金属丰度[Fe/H]、碳丰度[C/H])的综合数据集。
    \item \textbf{恒星参数预测:} 利用构建好的数据集,我们训练一个深度学习模型。该模型以预处理后的低分辨率光谱作为输入,学习光谱特征与恒星物理参数之间的复杂非线性关系,并最终用于预测海量未知恒星的$T_{\text{eff}}$、$\log g$、[Fe/H]和[C/H]。
    \item \textbf{CEMP候选体验证与筛选:} 在获得大规模恒星的预测参数后,我们依据表\ref{tab:star_criteria}中定义的CEMP恒星物理标准,对预测结果进行严格筛选,最终确定一个高置信度的CEMP恒星候选体列表。
\end{enumerate}


这一流程整合了大规模巡天数据与先进的机器学习技术,为在银河系中系统性地发掘稀有的贫金属恒星提供了有效途径。



\begin{table}[htbp]
    \centering
    \caption{恒星分类参数标准}

    
    \label{tab:star_criteria}
    \begin{tabular}{|l|l|}
    \hline
    \textbf{恒星类别} & \textbf{参数标准} \\
    \hline
    CEMP(碳增强金属贫星) & $[\mathrm{Fe}/\mathrm{H}] < -2.0$ 且 $[\mathrm{C}/\mathrm{Fe}] > 0.7$ \\
    \hline
    EMP(极贫金属星) & $[\mathrm{Fe}/\mathrm{H}] < -3$ \\
    \hline
    VMP(甚贫金属星) & $-2.5 \le [\mathrm{Fe}/\mathrm{H}] < -1.5$ \\
    \hline
    \end{tabular}
\end{table}

\subsection{波长范围选择依据}

本研究针对CEMP恒星搜索任务,精心选择了3800-8800 Å 的宽波段光谱数据。这一选择旨在全面覆盖对恒星大气物理参数($T_{\text{eff}}$, $\log g$, [Fe/H], [C/H])敏感的各类关键光谱特征,从而确保后续参数估计的精度与可靠性。该波段的选择依据与各科学参数的测量需求密切相关。

该波长范围不仅受益于地球大气较高的透过率,更重要的是其内部包含了从蓝端到近红外的一系列重要诊断特征:

\begin{itemize}
    \item \textbf{有效温度 ($T_{\text{eff}}$) 测量}: 光谱的整体形状,特别是蓝端的巴尔末跳变(Balmer Jump,约4000 Å附近)和氢的巴尔末线系(如H$\beta$, H$\gamma$),对恒星有效温度高度敏感。这些特征能够有效反映恒星大气的整体热辐射特性。
    \item \textbf{表面重力 ($\log g$) 诊断}: 多个谱线特征可用于约束表面重力。例如,Ca II H\&K线(3933, 3968 Å)的线翼宽度、Mg I b三线系(约5170 Å)以及近红外的Ca II三线系(8498, 8542, 8662 Å)都对恒星表面重力十分敏感,是确定恒星演化阶段(如主序星、巨星)的重要诊断工具。
    \item \textbf{碳丰度 ([C/H]) 指标}: CH分子的G带(约4300 Å)是测量恒星碳丰度的最关键特征之一。对于CEMP恒星的识别,该区域的高质量光谱数据至关重要。
    \item \textbf{金属丰度 ([Fe/H]) 指标}: 在5000-6000 Å 区域分布着密集的Fe I和Fe II谱线,它们是精确测定恒星铁丰度的基础。此外,其他元素的谱线(如Mg, Na)也为研究详细的化学丰度提供了信息。
\end{itemize}

综合来看,选择3800-8800 Å 的光谱数据,能够覆盖所有目标科学参数的关键诊断特征,为CEMP恒星的有效识别和详细物理参数测量提供全面且互补的信息。

\subsection{训练数据集构建与预处理}
\label{sec:data_preparation}
深度学习模型的性能高度依赖于训练数据的质量。为了构建一个适用于CEMP恒星参数估计的高质量参考数据集,我们整合了来自不同巡天项目的数据,并设计了一套系统性的数据预处理流程。

首先,我们构建用于模型训练的标签集。本研究使用的低分辨率光谱主要来源于LAMOST巡天。为了获得这些光谱对应的精确恒星物理参数(即标签),我们将LAMOST星表与多个高分辨率光谱巡天(如APOGEE DR17、LAMOST-Subaru、SAGA等)的星表进行交叉匹配。这一匹配过程确保了我们的训练样本同时拥有低分辨率光谱和高精度的参数标签($T_{\text{eff}}$, $\log g$, [Fe/H], [C/H])。

在构建了交叉匹配样本库之后,我们进一步对数据集进行筛选与优化以构造最终的训练集。第一步,为保证数据质量,我们移除了信噪比(SNR)低于50的光谱,确保了训练数据的可靠性。第二步,针对恒星金属丰度([Fe/H])存在的长尾分布问题,我们采用了随机均值采样方法。该方法通过平衡不同金属丰度区间的样本数量,有效缓解了因贫金属星样本稀少导致的数据不平衡问题,避免模型性能受到影响。

完成数据集的构造后,我们执行以下数据预处理步骤,旨在消除仪器效应和观测噪声,并将数据转换为适用于深度学习模型的标准化格式:

\begin{enumerate}
    \item \textbf{数据提取与格式转换:} 原始光谱数据以FITS格式存储。我们首先从中提取光谱流量、波长序列及恒星标识符,并将其整合为单一的CSV文件,以便于后续的高效处理。
    \item \textbf{静止系波长校正:} 恒星的径向速度会导致光谱产生多普勒频移。我们利用LAMOST提供的红移(z)值,将观测波长($\lambda_{\text{obs}}$)校正至静止波长($\lambda_{\text{rest}}$),其关系为 $\lambda_{\text{rest}} = \lambda_{\text{obs}} / (1+z)$。此步骤确保了不同光谱特征在共同的参考系下对齐,是后续分析的基础。
    \item \textbf{数据清洗与归一化:} 为保证光谱的物理真实性,我们采用中值滤波等方法修复由宇宙线或仪器故障造成的坏点和噪声。随后,对光谱进行连续谱归一化,即通过拟合光谱包络并用原始流量除以该包络,以消除距离、星际红化等外部因素的影响,凸显由恒星自身物理性质决定的吸收线特征。
    \item \textbf{光谱重采样:} 不同来源的光谱具有不同的分辨率和采样间隔。为了让模型能够处理统一尺寸的输入,我们将所有归一化后的光谱通过线性插值重采样至1 Å的等宽波长单元中。对于本研究使用的3800-8800 Å波段,每条光谱被转换为一个固定长度的向量。
\end{enumerate}

经过以上流程,我们最终获得了一个标准化的、清洁的、可直接用于模型训练的数据集。该数据集中的每一个样本都是一个代表恒星光谱特征的向量,并附有高精度的物理参数标签,为后续精准预测恒星参数提供了高质量的输入。

\section{多尺度注意力卷积循环网络模型}
\label{sec:model_architecture}

为解决现有模型在光谱特征提取中的局限性(\ref{subsec:motivation_dl}),我们提出了一种新颖的端到端深度学习架构,名为多尺度注意力卷积循环网络(Multi-scale Attentional Convolutional Recurrent Network)。该模型通过融合多尺度卷积、通道注意力机制和循环神经网络,旨在实现对恒星光谱信息更深层次、更全面的理解和建模。

\subsection{模型总体架构}
本模型的核心设计思想是分层、分模块地处理光谱数据。首先,通过一个金字塔形的卷积结构并行提取不同尺度的光谱特征;然后,利用通道注意力机制对这些特征进行加权,使模型能够聚焦于对当前任务最重要的信息;最后,通过一个长短期记忆网络(LSTM)捕捉光谱的全局序列依赖关系,并将最终的特征表示映射到物理参数空间。

\subsection{核心模块详解}

\subsubsection{金字塔形多尺度卷积模块}
与使用单一尺寸卷积核的传统CNN不同,我们的模型采用了一种金字塔形的并行卷积结构。该模块包含多个并行的卷积分支,每个分支使用不同大小的一维卷积核(例如,3x1, 5x1, 11x1等)。
\begin{itemize}
    \item \textbf{小卷积核}:负责捕捉光谱中尖锐、局部的吸收线细节。
    \item \textbf{大卷积核}:负责捕捉宽广的谱线翼部、分子带以及连续谱的缓变形状。
\end{itemize}
所有分支的输出特征图(Feature Maps)在通道维度上被拼接(Concatenate)在一起,形成一个包含了从局部到全局的多尺度特征表示。这种金字塔结构确保了模型不会在提取某一类特征时丢失另一类重要信息。

\subsubsection{通道注意力机制 (Channel Attention)}
在多尺度特征融合之后,我们引入了通道注意力模块。光谱的不同波段(即特征图的不同通道)对于预测不同的恒星参数其重要性是不同的。例如,CH G带对[C/Fe]的预测至关重要,而巴尔末线对$T_{\text{eff}}$的预测贡献更大。

通道注意力机制能够让模型自适应地学习每个特征通道的重要性权重。它首先对特征图进行全局平均池化,得到一个通道描述符向量,然后通过两个全连接层学习通道间的非线性依赖关系,最终为每个通道生成一个0到1之间的权重。这些权重将被乘回到原始的特征图上,从而增强关键特征通道的表达,抑制无关或噪声特征通道的影响。

\subsubsection{长短期记忆网络 (LSTM) 用于序列建模}
卷积操作本质上是局部的。为了捕捉光谱作为一个完整序列的长程依赖关系(例如,不同元素的多条谱线之间的相关性),我们将经过注意力加权的特征图输入到一个双向长短期记忆网络(Bidirectional LSTM)中。
\begin{itemize}
    \item \textbf{LSTM单元}:通过其独特的门控机制(输入门、遗忘门、输出门),LSTM能够有效地学习和记忆序列中的长期信息,解决了标准RNN中的梯度消失问题。
    \item \textbf{双向结构}:通过同时从前向和后向处理光谱序列,模型能够利用每个点的上下文信息,从而更全面地理解谱线特征的全局意义。
\end{itemize}

\subsection{输出与训练}
最终,LSTM层的输出被送入一个全连接层,回归得到四个恒星物理参数($T_{\text{eff}}$, $\log g$, [Fe/H], [C/H])。我们同样采用Adam优化器和均方误差(MSE)损失函数进行端到端的模型训练。
\section{Results}

This is the results section.
\section{Discussion}

This is the discussion section.
\section{Conclusion}

This is the conclusion section.
\section{Acknowledgments}

This is the acknowledgments section.
\section{Appendix}

This is the appendix section.


% \bibliography{sample7}{}
% \bibliographystyle{aasjournalv7}

%% This command is needed to show the entire author+affiliation list when
%% the collaboration and author truncation commands are used.  It has to
%% go at the end of the manuscript.
%\allauthors

%% Include this line if you are using the \added, \replaced, \deleted
%% commands to see a summary list of all changes at the end of the article.
%\listofchanges

\end{document}

% End of file `sample7.tex'.
