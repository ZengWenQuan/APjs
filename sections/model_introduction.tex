\section{多尺度注意力卷积循环网络模型}
\label{sec:model_architecture}

为解决现有模型在光谱特征提取中的局限性(\ref{subsec:motivation_dl}),我们提出了一种新颖的端到端深度学习架构,名为多尺度注意力卷积循环网络(Multi-scale Attentional Convolutional Recurrent Network)。该模型通过融合多尺度卷积、通道注意力机制和循环神经网络,旨在实现对恒星光谱信息更深层次、更全面的理解和建模。

\subsection{模型总体架构}
本模型的核心设计思想是分层、分模块地处理光谱数据。首先,通过一个金字塔形的卷积结构并行提取不同尺度的光谱特征;然后,利用通道注意力机制对这些特征进行加权,使模型能够聚焦于对当前任务最重要的信息;最后,通过一个长短期记忆网络(LSTM)捕捉光谱的全局序列依赖关系,并将最终的特征表示映射到物理参数空间。

\subsection{核心模块详解}

\subsubsection{金字塔形多尺度卷积模块}
与使用单一尺寸卷积核的传统CNN不同,我们的模型采用了一种金字塔形的并行卷积结构。该模块包含多个并行的卷积分支,每个分支使用不同大小的一维卷积核(例如,3x1, 5x1, 11x1等)。
\begin{itemize}
    \item \textbf{小卷积核}:负责捕捉光谱中尖锐、局部的吸收线细节。
    \item \textbf{大卷积核}:负责捕捉宽广的谱线翼部、分子带以及连续谱的缓变形状。
\end{itemize}
所有分支的输出特征图(Feature Maps)在通道维度上被拼接(Concatenate)在一起,形成一个包含了从局部到全局的多尺度特征表示。这种金字塔结构确保了模型不会在提取某一类特征时丢失另一类重要信息。

\subsubsection{通道注意力机制 (Channel Attention)}
在多尺度特征融合之后,我们引入了通道注意力模块。光谱的不同波段(即特征图的不同通道)对于预测不同的恒星参数其重要性是不同的。例如,CH G带对[C/Fe]的预测至关重要,而巴尔末线对$T_{\text{eff}}$的预测贡献更大。

通道注意力机制能够让模型自适应地学习每个特征通道的重要性权重。它首先对特征图进行全局平均池化,得到一个通道描述符向量,然后通过两个全连接层学习通道间的非线性依赖关系,最终为每个通道生成一个0到1之间的权重。这些权重将被乘回到原始的特征图上,从而增强关键特征通道的表达,抑制无关或噪声特征通道的影响。

\subsubsection{长短期记忆网络 (LSTM) 用于序列建模}
卷积操作本质上是局部的。为了捕捉光谱作为一个完整序列的长程依赖关系(例如,不同元素的多条谱线之间的相关性),我们将经过注意力加权的特征图输入到一个双向长短期记忆网络(Bidirectional LSTM)中。
\begin{itemize}
    \item \textbf{LSTM单元}:通过其独特的门控机制(输入门、遗忘门、输出门),LSTM能够有效地学习和记忆序列中的长期信息,解决了标准RNN中的梯度消失问题。
    \item \textbf{双向结构}:通过同时从前向和后向处理光谱序列,模型能够利用每个点的上下文信息,从而更全面地理解谱线特征的全局意义。
\end{itemize}

\subsection{输出与训练}
最终,LSTM层的输出被送入一个全连接层,回归得到四个恒星物理参数($T_{\text{eff}}$, $\log g$, [Fe/H], [C/H])。我们同样采用Adam优化器和均方误差(MSE)损失函数进行端到端的模型训练。