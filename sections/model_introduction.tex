\section{异构特征流网络 (HFS-Net)}
\label{sec:hfs_net_architecture}

为了更全面地从恒星光谱中提取多模态信息,我们提出了一种新颖的端到端深度学习架构,名为异构特征流网络 (Heterogeneous Feature Stream Network, HFS-Net)。该模型的核心思想是,针对物理来源与信号特性完全不同的连续谱和归一化谱,设计两个独立的、架构异构的特征提取分支,然后对提取到的深层特征进行高效融合,最终通过一个共享的预测头实现对多个恒星物理参数的同步回归。该模型的所有组件均采用模块化设计,可通过配置文件灵活替换与组合。

\subsection{模型总体架构}
\label{subsec:hfs_net_overview}

本模型的总体架构如图~\ref{fig:hfs_net_diagram}所示。数据处理流程遵循“分离-并行处理-融合-预测”的原则。首先,通过\ref{subsec:preprocessing}中描述的预处理方法将原始光谱分离为连续谱和归一化谱。随后,这两个信号被送入两个并行的、专门设计的特征提取分支:

\begin{itemize}
    \item \textbf{连续谱分支},一个基于频域的混合专家网络(MoE),旨在捕捉光谱的全局、低频特性。
    \item \textbf{归一化谱分支},一个基于多尺度卷积金字塔的网络,旨在捕捉吸收线的局部、高频细节。
\end{itemize}

两个分支提取的特征在经过维度对齐后,被一个融合模块进行合并。最终,融合后的特征向量被送入一个多任务预测头,以回归得到四个关键的恒星物理参数 ($T_{\text{eff}}$, $\log g$, [Fe/H], [C/Fe])。

\begin{figure}[h!]
    \centering
    % \includegraphics[width=\textwidth]{hfs_net_diagram.png} % 请将 hfs_net_diagram.png 替换为您的模型图文件名
    \framebox[0.9\textwidth][c]{\rule{0pt}{10cm} \large \textbf{HFS-Net 模型结构图占位符}}
    \caption{异构特征流网络 (HFS-Net) 的整体架构图。}
    \label{fig:hfs_net_diagram}
\end{figure}

\begin{table}[h!]
\centering
\caption{HFS-Net 模型关键模块的超参数配置。}
\label{tab:hfs_net_params}
\renewcommand{\arraystretch}{1.2} % 增加行高
\begin{tabular}{lll}
\toprule
\textbf{模块 (Module)} & \textbf{参数 (Parameter)} & \textbf{值 (Value)} \\
\midrule
\multirow{5}{*}{连续谱分支 \\ (\texttt{CustomMoEBranch})} 
 & FFT: n\_fft & 512 \\
 & FFT: hop\_length & 128 \\
 & MoE: num\_experts & 4 \\
 & MoE: k (top-k) & 2 \\
 & 专家网络结构 & 3层金字塔CNN \\
\midrule
\multirow{3}{*}{归一化谱分支 \\ (\texttt{MultiScalePyramidBranch})} % multrow包问题,这里检测不到主文件引入了
 & 金字塔层数 & 3 \\
 & 输出通道数 & {[}16, 32, 64{]} \\
 & 并行卷积核尺寸 & {[}3, 5, 7{]} \\
\midrule
融合模块 & 融合策略 & add (逐元素相加) \\
\midrule
\multirow{3}{*}{预测头 \\ (\texttt{MultiTaskHead})} 
 & 卷积金字塔层数 & 4 \\
 & FFN层维度 & {[}1024, 256, 128{]} \\
 & Dropout & 0.3 \\
\bottomrule
\end{tabular}
\end{table}



\subsection{核心模块详解}
\label{subsec:hfs_net_modules}

\subsubsection{连续谱分支:基于频域的混合专家网络}
\textbf{设计动机:} 连续谱的形状复杂,其缓变的趋势中可能混合了由恒星自身物理过程、星际红化及仪器响应等多种因素叠加而成的宽带特征。单一的卷积结构难以完全解耦这些多样化的模式。我们采用混合专家网络(Mixture-of-Experts, MoE),旨在通过“分而治之”的策略,让不同的“专家”子网络学习处理不同的特征模式,从而在不显著增加总计算量的前提下,提升模型的容量和表达能力。

\textbf{实现细节:} 该分支 (\texttt{CustomMoEBranch}) 首先通过快速傅里叶变换(FFT)将连续谱信号从波长域转换到频域。然后,一个轻量级的门控网络 (Gating Network) 会根据输入的频域特征,动态地、稀疏地选择激活多个专家网络中的Top-k个。每个专家网络都是一个独立的轻量级金字塔卷积模块,可以专注于学习和处理特定的频域模式(如特定的频率成分或形状)。

\subsubsection{归一化谱分支:多尺度金字塔卷积网络}
\textbf{设计动机:} 归一化谱中的吸收线形态各异,有的因恒星高速自转或高压环境而变得宽阔平缓,有的则因低温环境而显得尖锐狭窄。为了能同时、且鲁棒地捕捉这些不同形态的谱线特征,该分支 (\texttt{MultiScalePyramidBranch}) 采用了多尺度并行卷积的设计。

\textbf{实现细节:} 在金字塔结构的每一层,都并列着多个使用不同尺寸卷积核(例如,3x1, 5x1, 7x1)的卷积层。小卷积核擅长定位精确的、孤立的窄线轮廓,而大卷积核则能更好地整合宽线翼部或由多个邻近谱线混合(blending)而成的复合特征。通过将这些不同尺度的特征图进行拼接,模型能够构建一个对谱线形态变化不敏感的、信息丰富的层级化特征表示。

\subsubsection{特征融合与多任务预测}
\textbf{特征融合:} 两个分支的输出特征图在经过自适应池化层对齐长度,并通过1x1卷积对齐通道数后,通过简单的逐元素相加 (\texttt{strategy: 'add'}) 进行融合。这种融合策略简洁高效,它假设两个分支学习到的特征是相互对齐且可以线性叠加的,共同构成一个统一的、描述恒星状态的特征空间。

\textbf{多任务预测头:} 融合后的特征被送入一个共享的多任务预测头 (\texttt{MultiTaskHead})。该模块首先通过一个内部的卷积金字塔做最终的特征抽象,然后由一个多层前馈网络 (FFN) 回归出所有目标参数。将所有参数的预测放在同一个头中进行,是一种典型的多任务学习范式,它使得模型在学习预测一个参数时,可以利用到与其他参数共享的、潜在的物理关联信息(例如,$T_{\text{eff}}$和$\log g$之间存在的物理关联),从而可能提升所有任务的整体预测精度。

\subsection{训练策略}
\label{subsec:hfs_net_training}

我们采用均方误差(Mean Squared Error, MSE)作为模型的损失函数,并使用 AdamW 优化器进行端到端的模型训练。学习率调度器采用余弦退火(Cosine Annealing)策略。所有超参数均在配置文件 \texttt{flexiblefusionnet.yaml} 中定义。
