\section{Introduction}
CEMP(Carbon-Enhanced Metal-Poor)恒星是一类贫金属、富碳的老年恒星,其金属丰度([Fe/H])通常低于太阳值的千分之一,而碳丰度却显著增强。它们被认为是研究宇宙早期化学演化和恒星形成过程的关键天体。根据碳 enrich 水平和重元素特征,CEMP恒星被进一步划分为多个子类,如CEMP-s(s-process元素丰富)、CEMP-r(r-process元素丰富)和CEMP-no(氮和氧也贫化)。这些子类反映了不同的起源机制。

CEMP-s恒星约占CEMP恒星的80\%,多位于双星系统中,其碳和s-process元素 enrichment 可能源自伴星——一颗已经演化的AGB(渐近巨星支)恒星通过质量转移污染了原初气体 \cite{abia2002}。这一模型得到了观测到的双星轨道周期和自转速度的支持。

CEMP-no恒星则更为古老,可能形成于首批恒星(Population III)爆炸后的遗迹气体中。这些气体富含碳但缺乏 heavier metals,因此形成的第二代恒星继承了这种成分 \cite{christlieb2002}。例如,HE 1327-2326 是已知最metal-poor的CEMP-no恒星之一,其大气参数和化学组成为此假说提供了直接证据 \cite{ito2009}。

CEMP-r恒星含有较高的r-process元素(如铕Eu),这表明它们的形成环境受到了中子星并合或磁驱动超新星等事件的影响 \cite{hansen2011}。然而,这类恒星较为罕见,其确切起源仍需更多观测数据验证。

CEMP-no恒星多见于极低[Fe/H](< -3),如HE 1327-2326 ([Fe/H] ≈ -5.7, \cite{ito2009}) 和 SMSS J0313-6708 ([Fe/H] < -7.1, \cite{keller2014}),这表明它们可能形成于首批恒星(Population III)超新星爆炸后富含碳但贫metal的气体中。相比之下,CEMP-s恒星的[Fe/H]相对较高(≈-2.5至-3.0),这可能与双星系统中AGB伴星的质量转移有关,该过程需足够的时间积累,故对应稍高的metallicity (\cite{lucatello2005}).

CEMP-no恒星的[C/Fe]通常超过+1.0,且在极低[Fe/H]下保持高位,暗示其碳源可能来自高能爆发现象(如 faint supernovae)。而CEMP-s恒星的[C/Fe]虽同样高,但常伴随s-process元素(如Ba),其形成与AGB星的慢中子俘获过程相关,且在[Fe/H] > -3时更为常见 (\cite{abia2002}, \cite{bisterzo2011}).

CEMP-no恒星常表现为低[N/Fe]和高[C/N],与Population III恒星的混合模型一致 (\cite{norris2013}). CEMP-s恒星则多有[Ba/Fe] > +1.0,且[Ba/Eu]接近太阳能值,指示s-process主导 (\cite{masseron2010}). 少数CEMP-r恒星同时具有r-process元素(如Eu) enrichment,可能由中子星并合或磁驱动超新星引起 (\cite{hansen2011}).

观测数据显示,当[Fe/H] < -3时,CEMP恒星比例显著上升(约20\%),其中CEMP-no占主导 (\cite{yoon2016}). 这一趋势符合理论预测:早期宇宙中低metallicity气体更易在碳增强条件下形成低质量恒星 (\cite{frebel2015}).

CEMP恒星的研究不仅有助于理解早期宇宙的化学增丰历史,还为恒星形成和核合成理论提供了宝贵约束。未来高分辨率光谱观测和大样本统计将进一步揭示各类CEMP恒星的形成机制及其宇宙学意义。

我们对恒星的所有认知,从化学成分到物理状态,几乎都源于对其光谱的精细解读。理论上,恒星光谱由两部分组成:一个由其炽热表面发出的、平滑的**连续谱**作为背景;以及叠加其上的、由恒星外层大气中各种元素吸收特定波长光子而形成的**吸收线**。这些吸收线如同一张独特的“条形码”,其位置、深度和形状精确地揭示了恒星的化学丰度(如[Fe/H]和[C/Fe])、温度、压力乃至自转速度。然而,从观测数据中准确提取这些微弱的物理信号并非易事,需要仔细区分真实的吸收线与宇宙线、探测器噪声等干扰,这是恒星光谱分析中的关键挑战。

CEMP(Carbon-Enhanced Metal-Poor)恒星作为宇宙早期化学演化的关键探针,其发现与研究离不开大型天文巡天项目的推动。LAMOST(郭守敬望远镜)凭借其高效的光纤光谱观测能力,已成为发现和表征CEMP恒星的重要平台。自运行以来,LAMOST已释放数百万条恒星光谱数据,极大拓展了CEMP恒星样本量。例如,Li et al. (2015)\cite{li2015} 利用LAMOST数据鉴定出逾千颗CEMP候选星,显著提升了该类天体的统计基础。此外,结合Gaia的距离和自行数据及SDSS的测光信息,研究人员得以进一步解析CEMP恒星的大气参数和 kinematic 特性,深入探讨其起源机制(Yuan et al., 2020)\cite{yuan2020}。未来,LAMOST将持续助力CEMP恒星的系统性研究,为理解首批恒星核合成及银河系形成提供宝贵线索。

